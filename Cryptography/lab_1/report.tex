\documentclass[12pt]{article}

\usepackage{fullpage}
\usepackage{multicol,multirow}
\usepackage{tabularx}
\usepackage{ulem}
\usepackage[utf8]{inputenc}
\usepackage[russian]{babel}
\usepackage{color} %% это для отображения цвета в коде
\usepackage{listings} %% собственно, это и есть пакет listings

\usepackage{caption}
\parindent=1cm
\DeclareCaptionFont{white}{\color{white}} %% это сделает текст заголовка белым
%% код ниже нарисует серую рамочку вокруг заголовка кода.
\DeclareCaptionFormat{listing}{\colorbox{gray}{\parbox{\textwidth}{#1#2#3}}}
\captionsetup[lstlisting]{format=listing,labelfont=white,textfont=white}

\begin{document}

\section*{Лабораторная работа №\,1 по курсу Криптография}

Выполнила студентка группы 08-307 МАИ \textit{Усачева Елизавета}.

\subsection*{Задание}

Разложить каждое из чисел n1 и n2 на нетривиальные сомножители.

\subsubsection*{Вариант 0:}
\begin{lstlisting}
n1=284994967805859272853477327862245466978346919806585432133556769
959269315271111
n2=588334127500298760075185369594470830068532538546945053004260193 
945595259217406786089770982249290901568725608451508611541771305401
493870045176537915940249286141846560042142211542409633392281206603
637954450433183517707796541972311424681336480578342707287553065652
703205564979626673694073622920569912786167694946557747760150408914
919705012801904512377092934509169841743698528816250385854697524916
330387856905603362762198545848392715027208817212275244652775338978
6619
\end{lstlisting}


\subsection*{Введение}

Факторизация больших целых чисел является нетривиальной задачей. На вычислительной сложности задачи факторизации больших целых чисел основывается криптографический алгоритм с открытым ключом RSA, который стал первой системой, пригодной для шифрования и цифровой подписи. Трудоемкость алгоритмов, работающих с большими числами, оценивается количеством битовых операций. Зная количество операций, выполняемых компьютером за 1 секунду, можно оценить машинное время, необходимое для выполнения алгоритма. Оценка трудоемкости наиболее быстрых алгоритмов факторизации натуральных чисел имеет вид \textit{O(\exp $\sqrt{\log_{}n * \log_{}\log_{}n}$)}.

\subsection*{Метод решения}

\par Одним из самых быстрых и простых для реализации является так называемый метод квадратичного решета. Алгоритм пытается найти такие квадраты чисел, которые равны по модулю n (факторизуемое число), что часто приводит к факторизации n. Алгоритм работает в два этапа: этап сбора данных, где он собирает информацию, которая может привести к равенству квадратов и этап обработки данных, где он помещает всю собранную информацию в матрицу и обрабатывает её для получения равенства квадратов. Первый этап может быть легко распараллелен на много процессов, но второй этап требует большие объемы памяти и его трудно распараллелить.
Один из простых методов отыскания равных квадратов заключается в том, чтобы выбрать случайное число, возвести его в квадрат и надеяться, что остаток от деления на n является квадратом какого-либо другого числа.

\par Для нахождения множителей заданных чисел я решила воспользоваться утилитой \textit{msieve}, которая как раз содержит алгоритм квадратичного решета(SIQS), а также общий алгоритм решета числового поля(GNFS).

\begin{itemize}
    \item Сгенерированный \textit{msieve} лог факторизации первого числа \textit{n_1}:
    
\begin{lstlisting}
Wed Mar 04 16:49:40 2020  Msieve v. 1.53 (SVN 1005)
Wed Mar 04 16:49:40 2020  random seeds: 96097530 0fbb0776
Wed Mar 04 16:49:40 2020  factoring 28499496780585927285347732786
2245466978346919806585432133556769959269315271111 (78 digits)
Wed Mar 04 16:49:40 2020  searching for 15-digit factors
Wed Mar 04 16:49:40 2020  commencing quadratic sieve (78-digit input)
Wed Mar 04 16:49:40 2020  using multiplier of 7
Wed Mar 04 16:49:40 2020  using generic 32kb sieve core
Wed Mar 04 16:49:40 2020  sieve interval: 12 blocks of size 32768
Wed Mar 04 16:49:40 2020  processing polynomials in batches of 17
Wed Mar 04 16:49:40 2020  using a sieve bound of 996067 (39176 primes)
Wed Mar 04 16:49:40 2020  using large prime bound of 99606700 (26 bits)
Wed Mar 04 16:49:40 2020  using trial factoring cutoff of 27 bits
Wed Mar 04 16:49:40 2020  polynomial 'A' values have 10 factors
Wed Mar 04 16:51:52 2020  39688 relations (20266 full + 19422 combined
from 216084 partial), need 39272
Wed Mar 04 16:51:52 2020  begin with 236350 relations
Wed Mar 04 16:51:52 2020  reduce to 56754 relations in 2 passes
Wed Mar 04 16:51:52 2020  attempting to read 56754 relations
Wed Mar 04 16:51:52 2020  recovered 56754 relations
Wed Mar 04 16:51:52 2020  recovered 46081 polynomials
Wed Mar 04 16:51:52 2020  attempting to build 39688 cycles
Wed Mar 04 16:51:52 2020  found 39688 cycles in 1 passes
Wed Mar 04 16:51:52 2020  distribution of cycle lengths:
Wed Mar 04 16:51:52 2020     length 1 : 20266
Wed Mar 04 16:51:52 2020     length 2 : 19422
Wed Mar 04 16:51:52 2020  largest cycle: 2 relations
Wed Mar 04 16:51:53 2020  matrix is 39176 x 39688 (5.7 MB) with weight
1183174 (29.81/col)
Wed Mar 04 16:51:53 2020  sparse part has weight 1183174 (29.81/col)
Wed Mar 04 16:51:53 2020  filtering completed in 4 passes
Wed Mar 04 16:51:53 2020  matrix is 27699 x 27763 (4.3 MB) with weight
917229 (33.04/col)
Wed Mar 04 16:51:53 2020  sparse part has weight 917229 (33.04/col)
Wed Mar 04 16:51:53 2020  saving the first 48 matrix rows for later
Wed Mar 04 16:51:53 2020  matrix includes 64 packed rows
Wed Mar 04 16:51:53 2020  matrix is 27651 x 27763 (2.7 MB) with weight
641206 (23.10/col)
Wed Mar 04 16:51:53 2020  sparse part has weight 433368 (15.61/col)
Wed Mar 04 16:51:53 2020  commencing Lanczos iteration
Wed Mar 04 16:51:53 2020  memory use: 2.7 MB
Wed Mar 04 16:51:56 2020  lanczos halted after 439 iterations
(dim = 27649)
Wed Mar 04 16:51:56 2020  recovered 16 nontrivial dependencies
Wed Mar 04 16:51:57 2020  p39 factor:
397695326178862814397952263440193307813
Wed Mar 04 16:51:57 2020  p39 factor:
716616336792661370154476211778412420347
Wed Mar 04 16:51:57 2020  elapsed time 00:02:17
\end{lstlisting}

Чтобы найти делители второго числа необходимы очень большие вычислительные мощности. Поэтому, "поигравшись" с другими вариантами, найдя НОД с несколькими из них, мне удалось найти делители второго числа.

\subsection*{Ответ}
Сомножители первого числа:

\item 397695326178862814397952263440193307813
\item 716616336792661370154476211778412420347

Сомножители второго числа:

\item 1738981319876264319310040962007867376065418380910069311

5696916285733749398868672850488649953510095028267799834

086695894233669398059248857067719029306421047

\item 338321131328863932564818680296074563081532769076856609

725860402543148614331548252714137193761514829808776013

335385989807884956317300479543405309911233314952878929

489767681129617464167512042524219275287655118056649888

395988512738187341200216598691518616681800731333973241

210979683235278330374664946737167178077

\end{itemize}
    
\subsection*{Выводы}

В данной лабораторной работе я познакомилась с новой для меня областью на стыке математики и программирования - криптографией. В частности, изучила алгоритм шифрования RSA и различные алгоритмы факторизации больших чисел. В настоящее время задача факторизации некоторых больших чисел можно считать неразрешимой. 

\end{document}