\documentclass[dvipsnames,pdf, unicode, 12pt, a4paper, oneside, fleqn]{article}
\usepackage[utf8]{inputenc}
\usepackage[T2B]{fontenc}
\usepackage[english,russian]{babel}
\usepackage{hyperref}

\usepackage{listings}
\usepackage{longtable}
\oddsidemargin=-0.4mm
\textwidth=160mm
\topmargin=4.6mm
\textheight=210mm
\usepackage{geometry}
%% Страницы диссертациия должны иметь следующие поля:
%% левое --- 25 мм, правое --- 10 мм, верхнее --- 20 мм, нижнее --- 20 мм.
%% Абзацный отступ должен быть одинаковым по всему тексту и равен пяти знакам.
\geometry{
 a4paper,
 total={170mm,257mm},
 right=10mm,
 left=10mm,
 top=20mm,
 bottom=20mm,
}
\pagenumbering{gobble}

\usepackage{multicol}
\usepackage[]{amsmath}
\usepackage{multirow}

% THIS IS MY NEWLY DEFINED COMMAND
\newcommand\tline[2]{$\underset{\text{#1}}{\text{\underline{\hspace{#2}}}}$}

\usepackage{csquotes}
\DeclareQuoteStyle{russian}
    {\guillemotleft}{\guillemotright}[0.025em]
    {\quotedblbase}{\textquotedblleft}
\ExecuteQuoteOptions{style=russian}

\usepackage{longtable,array}

\newcolumntype{C}[1]{>{\centering\arraybackslash}p{#1}}
\setlength{\extrarowheight}{10pt}

\begin{document}

\begin{titlepage}
\begin{center}
\bfseries{\Large Министерство образования и науки\\Российской Федерации}

\vspace{12pt}

\bfseries{\Large Московский авиационный институт\\ (национальный исследовательский университет)}

\vspace{48pt}


%{\large Факультет информационных технологий и прикладной математики}

\vspace{36pt}


%{\large Кафедра вычислительной математики и~программирования}

\vspace{48pt}

{\huge ЖУРНАЛ}

\vspace{12pt}

{\large ПО ПРОИЗВОДСТВЕННОЙ ПРАКТИКЕ}


\end{center}

\vspace{72pt}

\begin{flushleft}
Наименование практики: {\itshape исследовательская}\\
Студент: Е.\,И. Усачева \\
Факультет №8, курс 3, группа 7 \\
\end{flushleft}

\vspace{12pt}

\begin{flushleft}
Практика с 29.06.20 по 12.07.20
\end{flushleft}

\vfill

\begin{center}
\bfseries Москва, \the\year
\end{center}
\end{titlepage}

\pagebreak

\begin{center}
\bfseries{\large ИНСТРУКЦИЯ }

\vspace{12pt}

\bfseries{о заполнении журнала по производственной практике}
\end{center}

\begin{multicols}{2}
{\small
Журнал по производственной практике студентов имеет единую форму для всех видов практик.

Задание в журнал вписывается руководителем практики от института в первые три-пять дней пребывания студентов на практике в соответствии с тематикой, утверждённой на кафедре до начала практики. Журнал по производственной практике является основным документом для текущего и итогового контроля выполнения заданий, требований инструкции и программы практики.

Табель прохождения практики, задание, а также технический отчёт выполняются каждым студентом самостоятельно.

Журнал заполняется студентом непрерывно в процессе прохождения всей практики и регулярно представляется для просмотра руководителям практики. Все их замечания подлежат немедленному выполнению.

В разделе «Табель прохождения практики» ежедневно должно быть указано, на каких рабочих местах и в качестве кого работал студент. Эти записи проверяются и заверяются цеховыми руководителями практики, в том числе мастерами и бригадирами. График прохождения практики заполняется в соответствии с графиком распределения студентов по рабочим местам практики, утверждённым руководителем предприятия.
В разделе «Рационализаторские предложения» должно быть приведено содержание поданных в цехе рационализаторских предложений со всеми необходимыми расчётами и эскизами. Рационализаторские предложения подаются индивидуально и коллективно.

Выполнение студентом задания по общественно-политической практике заносятся в раздел «Общественно-политическая практика». Выполнение работы по оказанию практической помощи предприятию (участие в выполнении спецзаданий, работа сверхурочно и т.п.) заносятся в раздел журнала «Работа в помощь предприятию» с последующим письменным подтверждением записанной работы соответствующими цеховыми руководителями.
Раздел «Технический отчёт по практике» должен быть заполнен особо тщательно. Записи необходимо делать чернилами в сжатой, но вместе с тем чёткой и ясной форме и технически грамотно. Студент обязан ежедневно подробно излагать содержание работы, выполняемой за каждый день. Содержание этого раздела должно отвечать тем конкретным требованиям, которые предъявляются к техническому отчёту заданием и программой практики. Технический отчёт должен показать умение студента критически оценивать работу данного производственного участка и отразить, в какой степени студент способен применить теоретические знания для решения конкретных производственных задач.

Иллюстративный и другие материалы, использованные студентом в других разделах журнала, в техническом отчёте не должны повторяться, следует ограничиваться лишь ссылкой на него. Участие студентов в производственно-технической конференции, выступление с докладами, рационализаторские предложения и т.п. должны заноситься на свободные страницы журнала.

{\bfseries Примечание.} Синьки, кальки и другие дополнения к журналу могут быть сделаны только с разрешения администрации предприятия и должны подшиваться в конце журнала.

Руководители практики от института обязаны следить за тем, чтобы каждый цеховой руководитель практики перед уходом студентов из данного цеха в другой цех вписывал в журнал студента отзывы об их работе в цехе.

Текущий контроль работы студентов осуществляется руководители практики от института и цеховыми руководителями практики заводов. Все замечания студентам руководители делают в письменном виде на страницах журнала, ставя при этом свою подпись и дату проверки.

Результаты защиты технического отчёта заносятся в протокол и одновременно заносятся в ведомость и зачётную книжку студента.

{\bfseries Примечание.} Нумерация чистых страниц журнала проставляется каждым студентом в своём журнале до начала практики.
}
\end{multicols}

\begin{center}
С инструкцией о заполнении журнала ознакомились:
\end{center}

«\hspace{0.5cm}» \tline{(дата)}{1.5in} \the\year\,г.\hfillСтудент Усачева Е.\,И. \tline{(подпись)}{1in}
\pagebreak

\begin{center}
\bfseries{\large ЗАДАНИЕ}
\end{center}

кафедры 806 по исследовательской практике: разработать прототип компьютерной 2D игры на игровом движке Unity.

\vspace*{\fill}
Руководитель практики от института:

\vspace{5pt}
\enquote{\hspace{0.5cm}} \tline{(дата)}{1.5in} \the\year\,г.\hfillКухтичев А.\,A. \tline{(подпись)}{1in}
\pagebreak
\begin{center}
\bfseries{\large ТАБЕЛЬ ПРОХОЖДЕНИЯ ПРАКТИКИ}
\end{center}

\begin{longtable}{|C{2cm}|C{6cm}|C{1.7cm}|C{1.5cm}|C{1.5cm}|C{2.8cm}|}
    \hline
    {\bfseries Дата} & {\bfseries Содержание или наименование проделанной работы} & {\bfseries Место работы} & \multicolumn{2}{c|}{{\bfseries Время работы}} & {\bfseries Подпись цехового руководителя}\\
    \cline {4-5} & & & Начало & Конец & \\
    \endfirsthead
    \hline
    {\bfseries Дата} & {\bfseries Содержание или наименование проделанной работы} & {\bfseries Место работы} & \multicolumn{2}{c|}{{\bfseries Время работы}} & {\bfseries Подпись цехового руководителя}\\
    \cline {4-5} & & & Начало & Конец & \\
    \hline
    \endhead
    \multicolumn{6}{c}{\textit{Продолжение на следующей странице}}
    \endfoot
    \endlastfoot
    \hline
    29.06.2020 & Получение задания & МАИ & 9:00 & 18:00 & \\
    \hline
    01.07.2020 & Выбор жанра игры & МАИ & 9:00 & 18:00 & \\
    \hline
    02.07.2020 & Написание сценария & МАИ & 9:00 & 18:00 & \\
    \hline
    03.07.2020 & Изучение материалов в сети Интернет & МАИ & 9:00 & 18:00 & \\
    \hline
    04.07.2020 & Установка необходимого ПО & МАИ & 9:00 & 18:00 & \\
    \hline
    05.07.2020 & Создание основной сцены + Анимация & МАИ & 9:00 & 18:00 & \\
    \hline
    06.07.2020 & Создание 1-го уровня & МАИ & 9:00 & 18:00 & \\
    \hline
    07.07.2020 & Создание персонажа и препятствий & МАИ & 9:00 & 18:00 & \\
    \hline
    09.07.2020 & Создание заключительных сцен & МАИ & 9:00 & 18:00 & \\
    \hline
    10.07.2020 & Отладка & МАИ & 9:00 & 18:00 & \\
    \hline
    11.07.2020 & Сборка проекта & МАИ & 9:00 & 18:00 & \\
    \hline
    12.07.2020 & Сдача журнала & МАИ & 9:00 & 18:00 &  \\
    \hline
\end{longtable}

\pagebreak

\begin{center}
\bfseries{\large Отзывы цеховых руководителей практики}
\end{center}
Студентка Усачева Е.\,И. разработала прототип компьютерной 2D игры для PC, Mac и Linux на игровом движке Unity.

Презентация защищена на комиссии кафедры 806. Работа выполнена в полном объёме. Рекомендую на оценку \enquote{\hspace{2cm}}. Все материалы сданы на кафедру.
\pagebreak


\begin{center}
\bfseries{\large ПРОТОКОЛ }

\vspace{12pt}

\bfseries{ЗАЩИТЫ ТЕХНИЧЕСКОГО ОТЧЁТА}
\end{center}
\noindent
по {\itshapeпроизводственной практике}

\vspace{8pt}
\noindent
студентами:
\noindent
Усачева Елизавета Игоревна

\begin{longtable}{p{7cm}|p{11cm}}
    \hline
    {\bfseries Слушали:} & {\bfseries Постановили:}  \\
    \endfirsthead
    \hline
    {\bfseries Слушали:} & {\bfseries Постановили:}  \\
    \hline
    \endhead
    \multicolumn{2}{c}{\textit{Продолжение на следующей странице}}
    \endfoot
    \endlastfoot
    Отчёт практиканта & считать практику выполненной и защищённой на\\
    \rule{0pt}{425pt} & Общая оценка: \underline{\hspace{2in}}\\
    \rule{0pt}{15pt} & \\
    \hline
\end{longtable}

\vfill

\noindent\begin{tabular}{@{}l l l}
Руководители: & Зайцев В.\,Е. & \underline{\hspace{2in}}\\
 \rule{0pt}{10pt} & Кухтичев А.\,А. & \underline{\hspace{2in}}
\end{tabular}
\vspace{12pt}

\noindent
Дата: 12 июля \the\year\,г.
\pagebreak

\begin{center}
\bfseries{\large МАТЕРИАЛЫ ПО РАЦИОНАЛИЗАТОРСКИМ ПРЕДЛОЖЕНИЯМ}
\end{center}
Так как был реализован лишь прототип игры, будущая ее реализация предплогает наличие многоуровнего игрового сценария. Также предлагается работа с графичиескими компонентами игры, например, создание 3D моделей главного персонажа и его целей (в нынешней реализации они представлены 2D изображениями).
\pagebreak

\begin{center}
\bfseries{\large ТЕХНИЧЕСКИЙ ОТЧЁТ ПО ПРАКТИКЕ}
\end{center}

\section*{Архитектура}
1 сцена: MenuScene - главная сцена с функцией запуска игрового уровня посредством тапа в любом месте экрана. \\
2 сцена: GameScene - основная игровая сцена, управление производится клавишами Left, Right, Up, Down на клавиатуре компьютера. \\
3 сцена: FailScene - сцена проигрыша. Запускается в случае проигрыша в сцене 2. Затем снова запускается сцена 2. \\
4 сцена: WinScene - сцена выиграша. Запускается в случае выигрыша в сцене 2. Затем игровое окно полностью завершает работу.
\section*{Описание}
Проект представляет из себя 2D компьютерную игру. Персонажем является фламинго, которому необходимо сесть на случайно появляющиеся коробки. При этом на него летят препятствия в виде камней, столкновение с которыми ведет к проигрышу. Процесс "присаживания" не реализован в силу ракурса и 2D визуализации, поэтому по факту реализовано просто наведение персонажа на целевой объект, который засчитывается как выигрышный. Управление персонажем осуществляется клавишами Left, Right, Up, Down на клавиатуре компьютера.
\section*{Реализация}
Для реализации был написан игровой сценарий, установлено необходимое ПО и изучены различные материалы по теме в сети Интернет. \\
ПО для реализации: Unity - игровой движок, редактор для создания игры; Visual Studio Code - текстовый редактор программного кода, скриптов (C#) для программирования поведения объектов и персонажа; Adobe Illustrator, Figma - графические редакторы для работы с векторными изображениями и прототипирования.
\section*{Тестирование}
Тестирование собранного проекта производилось на машинах под управлением ОС MacOS, непредсказуемого поведения или ошибок обнаружено не было.
\section*{Ссылка на GitHub}
\href{https://github.com/elfortitude/mai/tree/master/Summer_Practice}{https://github.com/elfortitude/mai/tree/master/Summer\_Practice}
\pagebreak

\end{document}
