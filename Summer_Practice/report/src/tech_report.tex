\begin{center}
\bfseries{\large ТЕХНИЧЕСКИЙ ОТЧЁТ ПО ПРАКТИКЕ}
\end{center}

\section*{Архитектура}
1 сцена: MenuScene - главная сцена с функцией запуска игрового уровня посредством тапа в любом месте экрана. \\
2 сцена: GameScene - основная игровая сцена, управление производится клавишами Left, Right, Up, Down на клавиатуре компьютера. \\
3 сцена: FailScene - сцена проигрыша. Запускается в случае проигрыша в сцене 2. Затем снова запускается сцена 2. \\
4 сцена: WinScene - сцена выиграша. Запускается в случае выигрыша в сцене 2. Затем игровое окно полностью завершает работу.
\section*{Описание}
Проект представляет из себя 2D компьютерную игру. Персонажем является фламинго, которому необходимо сесть на случайно появляющиеся коробки. При этом на него летят препятствия в виде камней, столкновение с которыми ведет к проигрышу. Процесс "присаживания" не реализован в силу ракурса и 2D визуализации, поэтому по факту реализовано просто наведение персонажа на целевой объект, который засчитывается как выигрышный. Управление персонажем осуществляется клавишами Left, Right, Up, Down на клавиатуре компьютера.
\section*{Реализация}
Для реализации был написан игровой сценарий, установлено необходимое ПО и изучены различные материалы по теме в сети Интернет. \\
ПО для реализации: Unity - игровой движок, редактор для создания игры; Visual Studio Code - текстовый редактор программного кода, скриптов (C#) для программирования поведения объектов и персонажа; Adobe Illustrator, Figma - графические редакторы для работы с векторными изображениями и прототипирования.
\section*{Тестирование}
Тестирование собранного проекта производилось на машинах под управлением ОС MacOS, непредсказуемого поведения или ошибок обнаружено не было.
\section*{Ссылка на GitHub}
\href{https://github.com/elfortitude/mai/tree/master/Summer_Practice}{https://github.com/elfortitude/mai/tree/master/Summer\_Practice}
\pagebreak